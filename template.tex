\documentclass[pdftex,english,oribibl]{llncs}

%% Spracheinstellungen laden
\usepackage[english]{babel}

%% Schriftart in der Ausgabe/Eingabe
\usepackage[T1]{fontenc}
\usepackage{textcomp}
\usepackage[latin1]{inputenc}

%% Zitate
\usepackage[numbers]{natbib}
\bibliographystyle{abbrvnat}
%\bibliographystyle{dinat}
%\bibliographystyle{plainnat}
%\bibliographystyle{splncs}
%% Similar to option "sectionbib" but \refname instead of \bibname
\makeatletter
\renewcommand\bibsection{\section*{\refname\@mkboth{\MakeUppercase{\refname}}{\MakeUppercase{\refname}}}}
\makeatother

%% Index
%\usepackage{makeidx}
%\makeindex

%% PDF Einstellungen
% muss nach natbib geladen werden!
\usepackage{nameref}
\usepackage{varioref}
\usepackage[pdfusetitle,pdftex,colorlinks]{hyperref}
\hypersetup{pdfborder={0 0 0}}
\hypersetup{bookmarksdepth=3}
\hypersetup{bookmarksopen=true}
\hypersetup{bookmarksopenlevel=1}
\hypersetup{bookmarksnumbered=true}
\usepackage{color}
\hypersetup{colorlinks=false}

%\usepackage[section]{tocbibind}

\makeatletter
\gdef\@keywords{}
\def\keywords#1{\gdef\@keywords{#1}}
\gdef\@subtitle{}
\def\subtitle#1{\gdef\@subtitle{#1}}

%% modified from llncs
\renewenvironment{abstract}{%
  \list{}{\advance\topsep by0.35cm\relax\small%
          \leftmargin=1cm%
          \labelwidth=\z@%
          \listparindent=\z@%
          \itemindent\listparindent%
          \rightmargin\leftmargin}%
          \item[\hskip\labelsep\bfseries\abstractname]}{%
  \if!\@keywords!\else{\item[~]\item[\hskip\labelsep\bfseries\keywordname]\@keywords}\fi%
  \endlist}

\AtBeginDocument{%
  \if!\@subtitle!\else\hypersetup{pdfsubject={\@subtitle}}\fi
  \if!\@keywords!\else\hypersetup{pdfkeywords={\@keywords}}\fi
}
\makeatother

% llncs hyperref fix
\makeatletter
\providecommand*{\toclevel@author}{0}
\providecommand*{\toclevel@title}{0}
\makeatother

%% Grafiken
\usepackage[pdftex]{graphicx}
\DeclareGraphicsExtensions{.pdf,.jpg,.png}
\usepackage{subfigure}

%% Mathe
\usepackage{amsmath}
\usepackage{amssymb}

%% Listings
\usepackage{listings}
\lstset{escapechar=\%, frame=tb, basicstyle=\footnotesize}

%% Sonstiges
\newcommand{\TODO}[1]{\par\textcolor{red}{#1}\marginpar{\textcolor{red}{TODO}}}
\newcommand{\TODOX}[1]{\textcolor{red}{#1}\marginpar{\textcolor{red}{TODO}}}
\pagestyle{plain}

% Keine "Schusterjungen"
\clubpenalty = 10000
% Keine "Hurenkinder"
\widowpenalty = 10000 \displaywidowpenalty = 10000

%%%%%%%%%%%%%%%%%%%%%%%%%%%%%%%%%%%%%%%%%%%%%%%%%%%%%%%%%%%%%%%%%%%%%%%%%%%%%%%
%%% BEGIN DOCUMENT
%%%%%%%%%%%%%%%%%%%%%%%%%%%%%%%%%%%%%%%%%%%%%%%%%%%%%%%%%%%%%%%%%%%%%%%%%%%%%%%
\title{My Title}
% \subtitle{My (optional) Subtitle}
\author{The Author}
\institute{University of Stuttgart\\Institute of Software Engineering (ISTE)\\70569 Stuttgart, Germany}


\begin{document}

\maketitle

\begin{abstract}
  This is my abstract.
\end{abstract}

\section{Introduction}

The rest of the paper is structured as follows:
First, we introduced

This is my introduction. \citet{Shaw2003WritingGoodSoftwareEngineeringResearchPapersMinitutorial} wrote a paper with hints on how to write good software engineering research papers. By the way, this was an example for using the \textit{natbib} command \texttt{\textbackslash{}citet\{\}}.

  Section~\ref{sec:anotherSection} presents everything one must know. The conclusions follow in Section~\ref{sec:conclusions}.

\section{Another Section}\label{sec:anotherSection}

  \textit{AspectJ} can be used to weave cross-cutting concerns into Java programs \citep{AspectJ2007}. By the way, this was an example for using the \textit{natbib} command \texttt{\textbackslash{}citep\{\}}.

  We will now demonstrate how to use subfigures (see Figure~\ref{fig:subfig}). Figure~\ref{fig:circle} shows a circle. A star is displayed in Figure~\ref{fig:star}.

  \begin{figure}
    \centering
    \subfigure[This is a circle.]{\label{fig:circle}
      \includegraphics[width=0.3\textwidth]{figures/template_circle.pdf}
    }
    \subfigure[This is a star.]{\label{fig:star}
      \includegraphics[width=0.3\textwidth]{figures/template_star.pdf}
    }
    \caption{A circle (a) and a star (b). Note that any caption ends with a full stop character.}
    \label{fig:subfig}
  \end{figure}

\section{Conclusions}\label{sec:conclusions}

  These are my conclusions.

\bibliography{template}

\end{document}
